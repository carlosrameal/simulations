% Options for packages loaded elsewhere
\PassOptionsToPackage{unicode}{hyperref}
\PassOptionsToPackage{hyphens}{url}
%
\documentclass[
]{article}
\usepackage{amsmath,amssymb}
\usepackage{iftex}
\ifPDFTeX
  \usepackage[T1]{fontenc}
  \usepackage[utf8]{inputenc}
  \usepackage{textcomp} % provide euro and other symbols
\else % if luatex or xetex
  \usepackage{unicode-math} % this also loads fontspec
  \defaultfontfeatures{Scale=MatchLowercase}
  \defaultfontfeatures[\rmfamily]{Ligatures=TeX,Scale=1}
\fi
\usepackage{lmodern}
\ifPDFTeX\else
  % xetex/luatex font selection
\fi
% Use upquote if available, for straight quotes in verbatim environments
\IfFileExists{upquote.sty}{\usepackage{upquote}}{}
\IfFileExists{microtype.sty}{% use microtype if available
  \usepackage[]{microtype}
  \UseMicrotypeSet[protrusion]{basicmath} % disable protrusion for tt fonts
}{}
\makeatletter
\@ifundefined{KOMAClassName}{% if non-KOMA class
  \IfFileExists{parskip.sty}{%
    \usepackage{parskip}
  }{% else
    \setlength{\parindent}{0pt}
    \setlength{\parskip}{6pt plus 2pt minus 1pt}}
}{% if KOMA class
  \KOMAoptions{parskip=half}}
\makeatother
\usepackage{xcolor}
\usepackage[margin=1in]{geometry}
\usepackage{color}
\usepackage{fancyvrb}
\newcommand{\VerbBar}{|}
\newcommand{\VERB}{\Verb[commandchars=\\\{\}]}
\DefineVerbatimEnvironment{Highlighting}{Verbatim}{commandchars=\\\{\}}
% Add ',fontsize=\small' for more characters per line
\usepackage{framed}
\definecolor{shadecolor}{RGB}{248,248,248}
\newenvironment{Shaded}{\begin{snugshade}}{\end{snugshade}}
\newcommand{\AlertTok}[1]{\textcolor[rgb]{0.94,0.16,0.16}{#1}}
\newcommand{\AnnotationTok}[1]{\textcolor[rgb]{0.56,0.35,0.01}{\textbf{\textit{#1}}}}
\newcommand{\AttributeTok}[1]{\textcolor[rgb]{0.13,0.29,0.53}{#1}}
\newcommand{\BaseNTok}[1]{\textcolor[rgb]{0.00,0.00,0.81}{#1}}
\newcommand{\BuiltInTok}[1]{#1}
\newcommand{\CharTok}[1]{\textcolor[rgb]{0.31,0.60,0.02}{#1}}
\newcommand{\CommentTok}[1]{\textcolor[rgb]{0.56,0.35,0.01}{\textit{#1}}}
\newcommand{\CommentVarTok}[1]{\textcolor[rgb]{0.56,0.35,0.01}{\textbf{\textit{#1}}}}
\newcommand{\ConstantTok}[1]{\textcolor[rgb]{0.56,0.35,0.01}{#1}}
\newcommand{\ControlFlowTok}[1]{\textcolor[rgb]{0.13,0.29,0.53}{\textbf{#1}}}
\newcommand{\DataTypeTok}[1]{\textcolor[rgb]{0.13,0.29,0.53}{#1}}
\newcommand{\DecValTok}[1]{\textcolor[rgb]{0.00,0.00,0.81}{#1}}
\newcommand{\DocumentationTok}[1]{\textcolor[rgb]{0.56,0.35,0.01}{\textbf{\textit{#1}}}}
\newcommand{\ErrorTok}[1]{\textcolor[rgb]{0.64,0.00,0.00}{\textbf{#1}}}
\newcommand{\ExtensionTok}[1]{#1}
\newcommand{\FloatTok}[1]{\textcolor[rgb]{0.00,0.00,0.81}{#1}}
\newcommand{\FunctionTok}[1]{\textcolor[rgb]{0.13,0.29,0.53}{\textbf{#1}}}
\newcommand{\ImportTok}[1]{#1}
\newcommand{\InformationTok}[1]{\textcolor[rgb]{0.56,0.35,0.01}{\textbf{\textit{#1}}}}
\newcommand{\KeywordTok}[1]{\textcolor[rgb]{0.13,0.29,0.53}{\textbf{#1}}}
\newcommand{\NormalTok}[1]{#1}
\newcommand{\OperatorTok}[1]{\textcolor[rgb]{0.81,0.36,0.00}{\textbf{#1}}}
\newcommand{\OtherTok}[1]{\textcolor[rgb]{0.56,0.35,0.01}{#1}}
\newcommand{\PreprocessorTok}[1]{\textcolor[rgb]{0.56,0.35,0.01}{\textit{#1}}}
\newcommand{\RegionMarkerTok}[1]{#1}
\newcommand{\SpecialCharTok}[1]{\textcolor[rgb]{0.81,0.36,0.00}{\textbf{#1}}}
\newcommand{\SpecialStringTok}[1]{\textcolor[rgb]{0.31,0.60,0.02}{#1}}
\newcommand{\StringTok}[1]{\textcolor[rgb]{0.31,0.60,0.02}{#1}}
\newcommand{\VariableTok}[1]{\textcolor[rgb]{0.00,0.00,0.00}{#1}}
\newcommand{\VerbatimStringTok}[1]{\textcolor[rgb]{0.31,0.60,0.02}{#1}}
\newcommand{\WarningTok}[1]{\textcolor[rgb]{0.56,0.35,0.01}{\textbf{\textit{#1}}}}
\usepackage{graphicx}
\makeatletter
\def\maxwidth{\ifdim\Gin@nat@width>\linewidth\linewidth\else\Gin@nat@width\fi}
\def\maxheight{\ifdim\Gin@nat@height>\textheight\textheight\else\Gin@nat@height\fi}
\makeatother
% Scale images if necessary, so that they will not overflow the page
% margins by default, and it is still possible to overwrite the defaults
% using explicit options in \includegraphics[width, height, ...]{}
\setkeys{Gin}{width=\maxwidth,height=\maxheight,keepaspectratio}
% Set default figure placement to htbp
\makeatletter
\def\fps@figure{htbp}
\makeatother
\setlength{\emergencystretch}{3em} % prevent overfull lines
\providecommand{\tightlist}{%
  \setlength{\itemsep}{0pt}\setlength{\parskip}{0pt}}
\setcounter{secnumdepth}{-\maxdimen} % remove section numbering
\ifLuaTeX
  \usepackage{selnolig}  % disable illegal ligatures
\fi
\usepackage{bookmark}
\IfFileExists{xurl.sty}{\usepackage{xurl}}{} % add URL line breaks if available
\urlstyle{same}
\hypersetup{
  pdftitle={Estimator Trials},
  hidelinks,
  pdfcreator={LaTeX via pandoc}}

\title{Estimator Trials}
\author{}
\date{\vspace{-2.5em}2025-03-25}

\begin{document}
\maketitle

\begin{Shaded}
\begin{Highlighting}[]
\NormalTok{jackknife\_po\_fn }\OtherTok{\textless{}{-}} \ControlFlowTok{function}\NormalTok{(data, indices) \{}
  \CommentTok{\# Leave{-}one{-}out data}
\NormalTok{  Y\_loo }\OtherTok{\textless{}{-}}\NormalTok{ data}\SpecialCharTok{$}\NormalTok{Y[indices]}
\NormalTok{  X\_loo }\OtherTok{\textless{}{-}} \FunctionTok{as.matrix}\NormalTok{(data[indices, }\SpecialCharTok{{-}}\DecValTok{1}\NormalTok{, }\AttributeTok{drop =} \ConstantTok{FALSE}\NormalTok{])}
  
  \CommentTok{\# First{-}step regressions (partialling out)}
\NormalTok{  mY\_loo }\OtherTok{\textless{}{-}} \FunctionTok{lm}\NormalTok{(Y\_loo }\SpecialCharTok{\textasciitilde{}}\NormalTok{ X\_loo[,}\SpecialCharTok{{-}}\DecValTok{1}\NormalTok{])}
\NormalTok{  mX1\_loo }\OtherTok{\textless{}{-}} \FunctionTok{lm}\NormalTok{(X\_loo[,}\DecValTok{1}\NormalTok{] }\SpecialCharTok{\textasciitilde{}}\NormalTok{ X\_loo[,}\SpecialCharTok{{-}}\DecValTok{1}\NormalTok{])}
  
  \CommentTok{\# Get residualized variables}
\NormalTok{  Y\_tilde\_loo }\OtherTok{\textless{}{-}}\NormalTok{ mY\_loo}\SpecialCharTok{$}\NormalTok{residuals}
\NormalTok{  X1\_tilde\_loo }\OtherTok{\textless{}{-}}\NormalTok{ mX1\_loo}\SpecialCharTok{$}\NormalTok{residuals}
  
  \CommentTok{\# Second{-}step regression}
\NormalTok{  m2\_loo }\OtherTok{\textless{}{-}} \FunctionTok{lm}\NormalTok{(Y\_tilde\_loo }\SpecialCharTok{\textasciitilde{}}\NormalTok{ X1\_tilde\_loo }\SpecialCharTok{{-}} \DecValTok{1}\NormalTok{)}
  
  \FunctionTok{return}\NormalTok{(}\FunctionTok{coef}\NormalTok{(m2\_loo)[}\DecValTok{1}\NormalTok{])  }\CommentTok{\# Return beta estimate}
\NormalTok{\}}
\end{Highlighting}
\end{Shaded}

\subsection{Linear model with correlated predictors (n=100,
rho=0.5)}\label{linear-model-with-correlated-predictors-n100-rho0.5}

\subsubsection{Case for p/n very high
(p/n=0.9)}\label{case-for-pn-very-high-pn0.9}

First we generate a population of n individuals with p regressors:

\begin{Shaded}
\begin{Highlighting}[]
\FunctionTok{set.seed}\NormalTok{(}\DecValTok{1}\NormalTok{)}
\NormalTok{n }\OtherTok{\textless{}{-}} \DecValTok{100}
\NormalTok{k }\OtherTok{\textless{}{-}} \FloatTok{1.1} \CommentTok{\# number of observations per coefficient}
\NormalTok{p }\OtherTok{\textless{}{-}} \FunctionTok{floor}\NormalTok{(n}\SpecialCharTok{/}\NormalTok{k) }\CommentTok{\#number of variables}
\NormalTok{beta }\OtherTok{\textless{}{-}} \FunctionTok{rep}\NormalTok{(}\FloatTok{0.5}\NormalTok{, p)}
\NormalTok{rho }\OtherTok{\textless{}{-}} \FloatTok{0.6} \CommentTok{\#correlation}
\NormalTok{Sigma }\OtherTok{\textless{}{-}} \FunctionTok{matrix}\NormalTok{(rho,p,p) }\CommentTok{\#correlation matrix}
\FunctionTok{diag}\NormalTok{(Sigma) }\OtherTok{\textless{}{-}} \DecValTok{1}  \CommentTok{\# Set diagonal to 1 for variances}
\NormalTok{mu }\OtherTok{\textless{}{-}} \FunctionTok{rep}\NormalTok{(}\DecValTok{0}\NormalTok{,p)}
\NormalTok{X }\OtherTok{\textless{}{-}} \FunctionTok{mvrnorm}\NormalTok{(n, mu, Sigma) }\CommentTok{\# simplified setting: all predictors are orthogonal }
\NormalTok{Y }\OtherTok{\textless{}{-}}\NormalTok{ X}\SpecialCharTok{\%*\%}\NormalTok{beta }\SpecialCharTok{+} \FunctionTok{rnorm}\NormalTok{(n)}
\end{Highlighting}
\end{Shaded}

We fit first an OLS model with intercept which yields quite bad results:

\begin{Shaded}
\begin{Highlighting}[]
\NormalTok{m1 }\OtherTok{\textless{}{-}} \FunctionTok{lm}\NormalTok{(Y }\SpecialCharTok{\textasciitilde{}}\NormalTok{ X)}
\FunctionTok{coef}\NormalTok{(m1)[}\DecValTok{1}\SpecialCharTok{:}\DecValTok{10}\NormalTok{]}
\end{Highlighting}
\end{Shaded}

\begin{verbatim}
## (Intercept)          X1          X2          X3          X4          X5 
##  -0.2926473   1.1864305   1.5901861   1.0055882   0.9358019   1.3114772 
##          X6          X7          X8          X9 
##  -0.3590001  -0.4728082   0.5839179   0.5560352
\end{verbatim}

We compute the estimate for X1 when partialling out (FWL theorem).

\begin{Shaded}
\begin{Highlighting}[]
\NormalTok{mY }\OtherTok{\textless{}{-}} \FunctionTok{lm}\NormalTok{(Y }\SpecialCharTok{\textasciitilde{}}\NormalTok{ X[,}\SpecialCharTok{{-}}\DecValTok{1}\NormalTok{])}
\NormalTok{mX1 }\OtherTok{\textless{}{-}} \FunctionTok{lm}\NormalTok{(X[,}\DecValTok{1}\NormalTok{] }\SpecialCharTok{\textasciitilde{}}\NormalTok{ X[,}\SpecialCharTok{{-}}\DecValTok{1}\NormalTok{])}
\NormalTok{Y.tilde }\OtherTok{\textless{}{-}}\NormalTok{ mY}\SpecialCharTok{$}\NormalTok{residuals}
\NormalTok{X1.tilde }\OtherTok{\textless{}{-}}\NormalTok{ mX1}\SpecialCharTok{$}\NormalTok{residuals}
\NormalTok{m2 }\OtherTok{\textless{}{-}} \FunctionTok{lm}\NormalTok{(Y.tilde }\SpecialCharTok{\textasciitilde{}}\NormalTok{ X1.tilde }\SpecialCharTok{{-}}\DecValTok{1}\NormalTok{ ) }\CommentTok{\# no intercept }
\end{Highlighting}
\end{Shaded}

OLS:

\begin{Shaded}
\begin{Highlighting}[]
\FunctionTok{coef}\NormalTok{(}\FunctionTok{summary}\NormalTok{(m1))[}\DecValTok{2}\NormalTok{,]}
\end{Highlighting}
\end{Shaded}

\begin{verbatim}
##   Estimate Std. Error    t value   Pr(>|t|) 
## 1.18643050 0.46585799 2.54676430 0.03136079
\end{verbatim}

Partialllng out:

\begin{Shaded}
\begin{Highlighting}[]
\FunctionTok{coef}\NormalTok{(}\FunctionTok{summary}\NormalTok{(m2))}
\end{Highlighting}
\end{Shaded}

\begin{verbatim}
##          Estimate Std. Error  t value     Pr(>|t|)
## X1.tilde  1.18643  0.1404615 8.446662 2.625596e-13
\end{verbatim}

Thanks to the FWL theorem we knew that the values of the estimates had
to be the same. And due to the ratio p/n being small, the estimate is
far from the real value of 0.5. We also see that the estimated S.E. is
different in each model, with OLS's being higher and therefore more
conservative although the real estimate would still be outside of the
CI.The S.E. in the PO model is not expected to work well when p/n is
very high as in this case.

\textbf{NOTE:} I think that there is a lot to consider here. In general,
even if p/n is small, you should not use the standard SE estimator
because of the homokedasticity assumption. The problem is that even if
your original model is homokedastic, nothing ensures you that the second
step PO model will also be. It would always be better to use the sanwich
estimator (Huber-White robust s.e.) or the jackknife estimator when
partialling out. I have the intuition that in this case in which every
regressor is independent to each other we do not introduce
homokedasticiy to the second model and therefore both the usual s.e. and
the Sandwich estimators behave very similarly. I should recheck this
when I introduce correlated regressors and possibly theoretically. Apart
from this, because p/n is very big, even robust s.e. will not behave
correctly as mentioned in the manual (refering to the sandwich
estimator: ``this standard error estimator formally works when p/n ≈ 0,
but fails in settings where p/n is not small'')

Now let's simulate the results for X1 a thousand times to get an idea of
the performance of the different s.e. estimators:

Due to jacknife being computationally heavy I will parallelize:

\begin{Shaded}
\begin{Highlighting}[]
\NormalTok{num\_cores }\OtherTok{\textless{}{-}} \FunctionTok{detectCores}\NormalTok{() }\SpecialCharTok{{-}} \DecValTok{1}  \CommentTok{\# Use available cores minus one}
\NormalTok{cl }\OtherTok{\textless{}{-}} \FunctionTok{makeCluster}\NormalTok{(num\_cores)}
\FunctionTok{registerDoParallel}\NormalTok{(cl)}

\NormalTok{N.sim }\OtherTok{\textless{}{-}} \DecValTok{1000}
\NormalTok{beta.LR }\OtherTok{\textless{}{-}}\NormalTok{ beta.PO }\OtherTok{\textless{}{-}}\NormalTok{ SE.LR }\OtherTok{\textless{}{-}}\NormalTok{ SE.PO }\OtherTok{\textless{}{-}}\NormalTok{ SE.POHW  }\OtherTok{\textless{}{-}}\NormalTok{ SE.LRHW }\OtherTok{\textless{}{-}}\NormalTok{ SE.LRJN }\OtherTok{\textless{}{-}}\NormalTok{ SE.POJN }\OtherTok{\textless{}{-}} \FunctionTok{c}\NormalTok{()}
\NormalTok{results }\OtherTok{\textless{}{-}} \FunctionTok{foreach}\NormalTok{(}\AttributeTok{i =} \DecValTok{1}\SpecialCharTok{:}\NormalTok{N.sim, }\AttributeTok{.combine =}\NormalTok{ rbind, }\AttributeTok{.packages =} \FunctionTok{c}\NormalTok{(}\StringTok{"sandwich"}\NormalTok{, }\StringTok{"boot"}\NormalTok{)) }\SpecialCharTok{\%dopar\%}\NormalTok{ \{}
  
\NormalTok{  beta }\OtherTok{\textless{}{-}} \FunctionTok{rep}\NormalTok{(}\FloatTok{0.5}\NormalTok{, p)}
\NormalTok{  X }\OtherTok{\textless{}{-}} \FunctionTok{replicate}\NormalTok{(p, }\FunctionTok{rnorm}\NormalTok{(n))}
\NormalTok{  Y }\OtherTok{\textless{}{-}}\NormalTok{ X }\SpecialCharTok{\%*\%}\NormalTok{ beta }\SpecialCharTok{+} \FunctionTok{rnorm}\NormalTok{(n)}
  
\NormalTok{  m1 }\OtherTok{\textless{}{-}} \FunctionTok{lm}\NormalTok{(Y }\SpecialCharTok{\textasciitilde{}}\NormalTok{ X)}
\NormalTok{  mY }\OtherTok{\textless{}{-}} \FunctionTok{lm}\NormalTok{(Y }\SpecialCharTok{\textasciitilde{}}\NormalTok{ X[,}\SpecialCharTok{{-}}\DecValTok{1}\NormalTok{])}
\NormalTok{  mX1 }\OtherTok{\textless{}{-}} \FunctionTok{lm}\NormalTok{(X[,}\DecValTok{1}\NormalTok{] }\SpecialCharTok{\textasciitilde{}}\NormalTok{ X[,}\SpecialCharTok{{-}}\DecValTok{1}\NormalTok{])}
\NormalTok{  Y.tilde }\OtherTok{\textless{}{-}}\NormalTok{ mY}\SpecialCharTok{$}\NormalTok{residuals}
\NormalTok{  X1.tilde }\OtherTok{\textless{}{-}}\NormalTok{ mX1}\SpecialCharTok{$}\NormalTok{residuals}
\NormalTok{  m2 }\OtherTok{\textless{}{-}} \FunctionTok{lm}\NormalTok{(Y.tilde }\SpecialCharTok{\textasciitilde{}}\NormalTok{ X1.tilde }\SpecialCharTok{{-}}\DecValTok{1}\NormalTok{)}
  
\NormalTok{  beta\_LR }\OtherTok{\textless{}{-}} \FunctionTok{coef}\NormalTok{(}\FunctionTok{summary}\NormalTok{(m1))[}\DecValTok{2}\NormalTok{, }\DecValTok{1}\NormalTok{]}
\NormalTok{  SE\_LR }\OtherTok{\textless{}{-}} \FunctionTok{coef}\NormalTok{(}\FunctionTok{summary}\NormalTok{(m1))[}\DecValTok{2}\NormalTok{, }\DecValTok{2}\NormalTok{]}
\NormalTok{  SE\_LRHW }\OtherTok{\textless{}{-}} \FunctionTok{sqrt}\NormalTok{(}\FunctionTok{diag}\NormalTok{(}\FunctionTok{vcovHC}\NormalTok{(m1, }\AttributeTok{type =} \StringTok{"HC0"}\NormalTok{))[}\DecValTok{2}\NormalTok{])}
\NormalTok{  SE\_LRJN }\OtherTok{\textless{}{-}} \FunctionTok{sqrt}\NormalTok{(}\FunctionTok{diag}\NormalTok{(}\FunctionTok{vcovHC}\NormalTok{(m1, }\AttributeTok{type =} \StringTok{"HC3"}\NormalTok{))[}\DecValTok{2}\NormalTok{])}
  
\NormalTok{  beta\_PO }\OtherTok{\textless{}{-}} \FunctionTok{coef}\NormalTok{(}\FunctionTok{summary}\NormalTok{(m2))[}\DecValTok{1}\NormalTok{]}
\NormalTok{  SE\_PO }\OtherTok{\textless{}{-}} \FunctionTok{coef}\NormalTok{(}\FunctionTok{summary}\NormalTok{(m2))[}\DecValTok{2}\NormalTok{]}
\NormalTok{  SE\_POHW }\OtherTok{\textless{}{-}} \FunctionTok{sqrt}\NormalTok{(}\FunctionTok{vcovHC}\NormalTok{(m2, }\AttributeTok{type =} \StringTok{"HC0"}\NormalTok{))}
  
\NormalTok{  data\_list }\OtherTok{\textless{}{-}} \FunctionTok{data.frame}\NormalTok{(}\AttributeTok{Y =}\NormalTok{ Y, X)}
\NormalTok{  jack\_res }\OtherTok{\textless{}{-}} \FunctionTok{boot}\NormalTok{(}\AttributeTok{data =}\NormalTok{ data\_list, }
                   \AttributeTok{statistic =}\NormalTok{ jackknife\_po\_fn, }
                   \AttributeTok{R =}\NormalTok{ n,  }
                   \AttributeTok{sim =} \StringTok{"ordinary"}\NormalTok{)}
\NormalTok{  SE\_POJN }\OtherTok{\textless{}{-}} \FunctionTok{sqrt}\NormalTok{((n }\SpecialCharTok{{-}} \DecValTok{1}\NormalTok{) }\SpecialCharTok{/}\NormalTok{ n }\SpecialCharTok{*} \FunctionTok{var}\NormalTok{(jack\_res}\SpecialCharTok{$}\NormalTok{t))}
  
  \FunctionTok{c}\NormalTok{(beta\_LR, SE\_LR, SE\_LRHW, SE\_LRJN, beta\_PO, SE\_PO, SE\_POHW, SE\_POJN)}
\NormalTok{\}}

\FunctionTok{stopCluster}\NormalTok{(cl)}

\CommentTok{\# Convert results to a data frame}
\NormalTok{results\_df }\OtherTok{\textless{}{-}} \FunctionTok{as.data.frame}\NormalTok{(results)}
\FunctionTok{colnames}\NormalTok{(results\_df) }\OtherTok{\textless{}{-}} \FunctionTok{c}\NormalTok{(}\StringTok{"beta.LR"}\NormalTok{, }\StringTok{"SE.LR"}\NormalTok{, }\StringTok{"SE.LRHW"}\NormalTok{, }\StringTok{"SE.LRJN"}\NormalTok{, }\StringTok{"beta.PO"}\NormalTok{, }\StringTok{"SE.PO"}\NormalTok{, }\StringTok{"SE.POHW"}\NormalTok{,  }\StringTok{"SE.POJN"}\NormalTok{)}
\end{Highlighting}
\end{Shaded}

\begin{Shaded}
\begin{Highlighting}[]
\NormalTok{mean\_table }\OtherTok{\textless{}{-}} \FunctionTok{data.frame}\NormalTok{(}
  \AttributeTok{Statistic =} \StringTok{"Mean"}\NormalTok{,}
  \FunctionTok{t}\NormalTok{(}\FunctionTok{colMeans}\NormalTok{(results\_df, }\AttributeTok{na.rm =} \ConstantTok{TRUE}\NormalTok{))  }\CommentTok{\# Transpose to make it a row}
\NormalTok{)}
\NormalTok{mean\_table}
\end{Highlighting}
\end{Shaded}

\begin{verbatim}
##   Statistic  beta.LR     SE.LR    SE.LRHW  SE.LRJN  beta.PO     SE.PO
## 1      Mean 0.502617 0.3335074 0.09885051 1.251345 0.502617 0.1005563
##      SE.POHW  SE.POJN
## 1 0.09885051 5.314561
\end{verbatim}

Indeed, we see that the usual OLS S.E. has a very good performance when
compared to the Monte Carlo estimator. The three S.E. computed through
the PO model are quite far from the actual one. The fact that they seem
to be consistently smaller is very problematic if we wanted to compute
confidence intervals.

\textbf{NOTE:} When performing the jackknife estimator when partialling
out we should first remove the correspondent observation for the
Leave-one-out and then compute both steps of the method. Applying HC3 to
the second regression does not really function as a jackknife for the PO
model.

\subsubsection{Case for p/n high
(p/n=0.5)}\label{case-for-pn-high-pn0.5}

We fit first an OLS model with intercept which yields somewhat better
results:

\begin{verbatim}
## (Intercept)          X1          X2          X3          X4          X5 
## -0.08920452  0.47609210  0.55626645  0.28389366  0.68435002  0.46856785 
##          X6          X7          X8          X9 
##  0.73464906  0.43395315  0.71626378  0.45615883
\end{verbatim}

We compute the estimate for X1 when partialling out (FWL theorem).

OLS:

\begin{verbatim}
##     Estimate   Std. Error      t value     Pr(>|t|) 
## 0.4760920994 0.1307507163 3.6412198170 0.0006533358
\end{verbatim}

Partialllng out:

\begin{verbatim}
##           Estimate Std. Error  t value     Pr(>|t|)
## X1.tilde 0.4760921 0.09198659 5.175669 1.191479e-06
\end{verbatim}

Now, due to the ratio p/n being bigger, the estimate is much closer to
the real value of 0.5. We continue to see that the estimated S.D. is
different in each model, with OLS's again being higher and therefore
more conservative. In this case both C.I. encompass the real value.

Let's run the simulations:

\begin{Shaded}
\begin{Highlighting}[]
\NormalTok{num\_cores }\OtherTok{\textless{}{-}} \FunctionTok{detectCores}\NormalTok{() }\SpecialCharTok{{-}} \DecValTok{1}  \CommentTok{\# Use available cores minus one}
\NormalTok{cl }\OtherTok{\textless{}{-}} \FunctionTok{makeCluster}\NormalTok{(num\_cores)}
\FunctionTok{registerDoParallel}\NormalTok{(cl)}

\NormalTok{N.sim }\OtherTok{\textless{}{-}} \DecValTok{1000}
\NormalTok{beta.LR }\OtherTok{\textless{}{-}}\NormalTok{ beta.PO }\OtherTok{\textless{}{-}}\NormalTok{ SE.LR }\OtherTok{\textless{}{-}}\NormalTok{ SE.PO }\OtherTok{\textless{}{-}}\NormalTok{ SE.POHW  }\OtherTok{\textless{}{-}}\NormalTok{ SE.LRHW }\OtherTok{\textless{}{-}}\NormalTok{ SE.LRJN }\OtherTok{\textless{}{-}}\NormalTok{ SE.POJN }\OtherTok{\textless{}{-}} \FunctionTok{c}\NormalTok{()}
\NormalTok{results }\OtherTok{\textless{}{-}} \FunctionTok{foreach}\NormalTok{(}\AttributeTok{i =} \DecValTok{1}\SpecialCharTok{:}\NormalTok{N.sim, }\AttributeTok{.combine =}\NormalTok{ rbind, }\AttributeTok{.packages =} \FunctionTok{c}\NormalTok{(}\StringTok{"sandwich"}\NormalTok{, }\StringTok{"boot"}\NormalTok{)) }\SpecialCharTok{\%dopar\%}\NormalTok{ \{}
  
\NormalTok{  beta }\OtherTok{\textless{}{-}} \FunctionTok{rep}\NormalTok{(}\FloatTok{0.5}\NormalTok{, p)}
\NormalTok{  X }\OtherTok{\textless{}{-}} \FunctionTok{replicate}\NormalTok{(p, }\FunctionTok{rnorm}\NormalTok{(n))}
\NormalTok{  Y }\OtherTok{\textless{}{-}}\NormalTok{ X }\SpecialCharTok{\%*\%}\NormalTok{ beta }\SpecialCharTok{+} \FunctionTok{rnorm}\NormalTok{(n)}
  
\NormalTok{  m1 }\OtherTok{\textless{}{-}} \FunctionTok{lm}\NormalTok{(Y }\SpecialCharTok{\textasciitilde{}}\NormalTok{ X)}
\NormalTok{  mY }\OtherTok{\textless{}{-}} \FunctionTok{lm}\NormalTok{(Y }\SpecialCharTok{\textasciitilde{}}\NormalTok{ X[,}\SpecialCharTok{{-}}\DecValTok{1}\NormalTok{])}
\NormalTok{  mX1 }\OtherTok{\textless{}{-}} \FunctionTok{lm}\NormalTok{(X[,}\DecValTok{1}\NormalTok{] }\SpecialCharTok{\textasciitilde{}}\NormalTok{ X[,}\SpecialCharTok{{-}}\DecValTok{1}\NormalTok{])}
\NormalTok{  Y.tilde }\OtherTok{\textless{}{-}}\NormalTok{ mY}\SpecialCharTok{$}\NormalTok{residuals}
\NormalTok{  X1.tilde }\OtherTok{\textless{}{-}}\NormalTok{ mX1}\SpecialCharTok{$}\NormalTok{residuals}
\NormalTok{  m2 }\OtherTok{\textless{}{-}} \FunctionTok{lm}\NormalTok{(Y.tilde }\SpecialCharTok{\textasciitilde{}}\NormalTok{ X1.tilde }\SpecialCharTok{{-}}\DecValTok{1}\NormalTok{)}
  
\NormalTok{  beta\_LR }\OtherTok{\textless{}{-}} \FunctionTok{coef}\NormalTok{(}\FunctionTok{summary}\NormalTok{(m1))[}\DecValTok{2}\NormalTok{, }\DecValTok{1}\NormalTok{]}
\NormalTok{  SE\_LR }\OtherTok{\textless{}{-}} \FunctionTok{coef}\NormalTok{(}\FunctionTok{summary}\NormalTok{(m1))[}\DecValTok{2}\NormalTok{, }\DecValTok{2}\NormalTok{]}
\NormalTok{  SE\_LRHW }\OtherTok{\textless{}{-}} \FunctionTok{sqrt}\NormalTok{(}\FunctionTok{diag}\NormalTok{(}\FunctionTok{vcovHC}\NormalTok{(m1, }\AttributeTok{type =} \StringTok{"HC0"}\NormalTok{))[}\DecValTok{2}\NormalTok{])}
\NormalTok{  SE\_LRJN }\OtherTok{\textless{}{-}} \FunctionTok{sqrt}\NormalTok{(}\FunctionTok{diag}\NormalTok{(}\FunctionTok{vcovHC}\NormalTok{(m1, }\AttributeTok{type =} \StringTok{"HC3"}\NormalTok{))[}\DecValTok{2}\NormalTok{])}
  
\NormalTok{  beta\_PO }\OtherTok{\textless{}{-}} \FunctionTok{coef}\NormalTok{(}\FunctionTok{summary}\NormalTok{(m2))[}\DecValTok{1}\NormalTok{]}
\NormalTok{  SE\_PO }\OtherTok{\textless{}{-}} \FunctionTok{coef}\NormalTok{(}\FunctionTok{summary}\NormalTok{(m2))[}\DecValTok{2}\NormalTok{]}
\NormalTok{  SE\_POHW }\OtherTok{\textless{}{-}} \FunctionTok{sqrt}\NormalTok{(}\FunctionTok{vcovHC}\NormalTok{(m2, }\AttributeTok{type =} \StringTok{"HC0"}\NormalTok{))}
  
\NormalTok{  data\_list }\OtherTok{\textless{}{-}} \FunctionTok{data.frame}\NormalTok{(}\AttributeTok{Y =}\NormalTok{ Y, X)}
\NormalTok{  jack\_res }\OtherTok{\textless{}{-}} \FunctionTok{boot}\NormalTok{(}\AttributeTok{data =}\NormalTok{ data\_list, }
                   \AttributeTok{statistic =}\NormalTok{ jackknife\_po\_fn, }
                   \AttributeTok{R =}\NormalTok{ n,  }
                   \AttributeTok{sim =} \StringTok{"ordinary"}\NormalTok{)}
\NormalTok{  SE\_POJN }\OtherTok{\textless{}{-}} \FunctionTok{sqrt}\NormalTok{((n }\SpecialCharTok{{-}} \DecValTok{1}\NormalTok{) }\SpecialCharTok{/}\NormalTok{ n }\SpecialCharTok{*} \FunctionTok{var}\NormalTok{(jack\_res}\SpecialCharTok{$}\NormalTok{t))}
  
  \FunctionTok{c}\NormalTok{(beta\_LR, SE\_LR, SE\_LRHW, SE\_LRJN, beta\_PO, SE\_PO, SE\_POHW, SE\_POJN)}
\NormalTok{\}}

\FunctionTok{stopCluster}\NormalTok{(cl)}

\CommentTok{\# Convert results to a data frame}
\NormalTok{results\_df }\OtherTok{\textless{}{-}} \FunctionTok{as.data.frame}\NormalTok{(results)}
\FunctionTok{colnames}\NormalTok{(results\_df) }\OtherTok{\textless{}{-}} \FunctionTok{c}\NormalTok{(}\StringTok{"beta.LR"}\NormalTok{, }\StringTok{"SE.LR"}\NormalTok{, }\StringTok{"SE.LRHW"}\NormalTok{, }\StringTok{"SE.LRJN"}\NormalTok{, }\StringTok{"beta.PO"}\NormalTok{, }\StringTok{"SE.PO"}\NormalTok{, }\StringTok{"SE.POHW"}\NormalTok{,  }\StringTok{"SE.POJN"}\NormalTok{)}
\end{Highlighting}
\end{Shaded}

\begin{verbatim}
##   Statistic   beta.LR     SE.LR    SE.LRHW   SE.LRJN   beta.PO     SE.PO
## 1      Mean 0.5020036 0.1417856 0.09800343 0.2061215 0.5020036 0.0997499
##      SE.POHW   SE.POJN
## 1 0.09800343 0.2741981
\end{verbatim}

The LR s.e. continues to greatly outperform the PO s.e. although this
time they are significantly closer. The three PO s.e. continue to behave
similarly, which to me points even more to homokedasticity due to how
the regressors were created.

\subsubsection{Case for p/n small
(p/n=0.05)}\label{case-for-pn-small-pn0.05}

We fit first an OLS model with intercept which yields much better
results:

\begin{verbatim}
## (Intercept)          X1          X2          X3          X4          X5 
## -0.02953446  0.38795776  0.48188243  0.47398551  0.40398547  0.44233768
\end{verbatim}

We compute the estimate for X1 when partialling out (FWL theorem).

OLS:

\begin{verbatim}
##     Estimate   Std. Error      t value     Pr(>|t|) 
## 0.3879577610 0.1107560842 3.5028121841 0.0007065748
\end{verbatim}

Partialllng out:

\begin{verbatim}
##           Estimate Std. Error  t value     Pr(>|t|)
## X1.tilde 0.3879578   0.107923 3.594765 0.0005082119
\end{verbatim}

This time in particular X1's estimate is the worse out of the 5. It is
worse than for the previous case where p/n=0.5, although in general the
model worked better now. The SD of normal OLS and Partialling out now
are very similar and the real value is inside the C.I.

\begin{Shaded}
\begin{Highlighting}[]
\NormalTok{num\_cores }\OtherTok{\textless{}{-}} \FunctionTok{detectCores}\NormalTok{() }\SpecialCharTok{{-}} \DecValTok{1}  \CommentTok{\# Use available cores minus one}
\NormalTok{cl }\OtherTok{\textless{}{-}} \FunctionTok{makeCluster}\NormalTok{(num\_cores)}
\FunctionTok{registerDoParallel}\NormalTok{(cl)}

\NormalTok{N.sim }\OtherTok{\textless{}{-}} \DecValTok{1000}
\NormalTok{beta.LR }\OtherTok{\textless{}{-}}\NormalTok{ beta.PO }\OtherTok{\textless{}{-}}\NormalTok{ SE.LR }\OtherTok{\textless{}{-}}\NormalTok{ SE.PO }\OtherTok{\textless{}{-}}\NormalTok{ SE.POHW  }\OtherTok{\textless{}{-}}\NormalTok{ SE.LRHW }\OtherTok{\textless{}{-}}\NormalTok{ SE.LRJN }\OtherTok{\textless{}{-}}\NormalTok{ SE.POJN }\OtherTok{\textless{}{-}} \FunctionTok{c}\NormalTok{()}
\NormalTok{results }\OtherTok{\textless{}{-}} \FunctionTok{foreach}\NormalTok{(}\AttributeTok{i =} \DecValTok{1}\SpecialCharTok{:}\NormalTok{N.sim, }\AttributeTok{.combine =}\NormalTok{ rbind, }\AttributeTok{.packages =} \FunctionTok{c}\NormalTok{(}\StringTok{"sandwich"}\NormalTok{, }\StringTok{"boot"}\NormalTok{)) }\SpecialCharTok{\%dopar\%}\NormalTok{ \{}
  
\NormalTok{  beta }\OtherTok{\textless{}{-}} \FunctionTok{rep}\NormalTok{(}\FloatTok{0.5}\NormalTok{, p)}
\NormalTok{  X }\OtherTok{\textless{}{-}} \FunctionTok{replicate}\NormalTok{(p, }\FunctionTok{rnorm}\NormalTok{(n))}
\NormalTok{  Y }\OtherTok{\textless{}{-}}\NormalTok{ X }\SpecialCharTok{\%*\%}\NormalTok{ beta }\SpecialCharTok{+} \FunctionTok{rnorm}\NormalTok{(n)}
  
\NormalTok{  m1 }\OtherTok{\textless{}{-}} \FunctionTok{lm}\NormalTok{(Y }\SpecialCharTok{\textasciitilde{}}\NormalTok{ X)}
\NormalTok{  mY }\OtherTok{\textless{}{-}} \FunctionTok{lm}\NormalTok{(Y }\SpecialCharTok{\textasciitilde{}}\NormalTok{ X[,}\SpecialCharTok{{-}}\DecValTok{1}\NormalTok{])}
\NormalTok{  mX1 }\OtherTok{\textless{}{-}} \FunctionTok{lm}\NormalTok{(X[,}\DecValTok{1}\NormalTok{] }\SpecialCharTok{\textasciitilde{}}\NormalTok{ X[,}\SpecialCharTok{{-}}\DecValTok{1}\NormalTok{])}
\NormalTok{  Y.tilde }\OtherTok{\textless{}{-}}\NormalTok{ mY}\SpecialCharTok{$}\NormalTok{residuals}
\NormalTok{  X1.tilde }\OtherTok{\textless{}{-}}\NormalTok{ mX1}\SpecialCharTok{$}\NormalTok{residuals}
\NormalTok{  m2 }\OtherTok{\textless{}{-}} \FunctionTok{lm}\NormalTok{(Y.tilde }\SpecialCharTok{\textasciitilde{}}\NormalTok{ X1.tilde }\SpecialCharTok{{-}}\DecValTok{1}\NormalTok{)}
  
\NormalTok{  beta\_LR }\OtherTok{\textless{}{-}} \FunctionTok{coef}\NormalTok{(}\FunctionTok{summary}\NormalTok{(m1))[}\DecValTok{2}\NormalTok{, }\DecValTok{1}\NormalTok{]}
\NormalTok{  SE\_LR }\OtherTok{\textless{}{-}} \FunctionTok{coef}\NormalTok{(}\FunctionTok{summary}\NormalTok{(m1))[}\DecValTok{2}\NormalTok{, }\DecValTok{2}\NormalTok{]}
\NormalTok{  SE\_LRHW }\OtherTok{\textless{}{-}} \FunctionTok{sqrt}\NormalTok{(}\FunctionTok{diag}\NormalTok{(}\FunctionTok{vcovHC}\NormalTok{(m1, }\AttributeTok{type =} \StringTok{"HC0"}\NormalTok{))[}\DecValTok{2}\NormalTok{])}
\NormalTok{  SE\_LRJN }\OtherTok{\textless{}{-}} \FunctionTok{sqrt}\NormalTok{(}\FunctionTok{diag}\NormalTok{(}\FunctionTok{vcovHC}\NormalTok{(m1, }\AttributeTok{type =} \StringTok{"HC3"}\NormalTok{))[}\DecValTok{2}\NormalTok{])}
  
\NormalTok{  beta\_PO }\OtherTok{\textless{}{-}} \FunctionTok{coef}\NormalTok{(}\FunctionTok{summary}\NormalTok{(m2))[}\DecValTok{1}\NormalTok{]}
\NormalTok{  SE\_PO }\OtherTok{\textless{}{-}} \FunctionTok{coef}\NormalTok{(}\FunctionTok{summary}\NormalTok{(m2))[}\DecValTok{2}\NormalTok{]}
\NormalTok{  SE\_POHW }\OtherTok{\textless{}{-}} \FunctionTok{sqrt}\NormalTok{(}\FunctionTok{vcovHC}\NormalTok{(m2, }\AttributeTok{type =} \StringTok{"HC0"}\NormalTok{))}
  
\NormalTok{  data\_list }\OtherTok{\textless{}{-}} \FunctionTok{data.frame}\NormalTok{(}\AttributeTok{Y =}\NormalTok{ Y, X)}
\NormalTok{  jack\_res }\OtherTok{\textless{}{-}} \FunctionTok{boot}\NormalTok{(}\AttributeTok{data =}\NormalTok{ data\_list, }
                   \AttributeTok{statistic =}\NormalTok{ jackknife\_po\_fn, }
                   \AttributeTok{R =}\NormalTok{ n,  }
                   \AttributeTok{sim =} \StringTok{"ordinary"}\NormalTok{)}
\NormalTok{  SE\_POJN }\OtherTok{\textless{}{-}} \FunctionTok{sqrt}\NormalTok{((n }\SpecialCharTok{{-}} \DecValTok{1}\NormalTok{) }\SpecialCharTok{/}\NormalTok{ n }\SpecialCharTok{*} \FunctionTok{var}\NormalTok{(jack\_res}\SpecialCharTok{$}\NormalTok{t))}
  
  \FunctionTok{c}\NormalTok{(beta\_LR, SE\_LR, SE\_LRHW, SE\_LRJN, beta\_PO, SE\_PO, SE\_POHW, SE\_POJN)}
\NormalTok{\}}

\FunctionTok{stopCluster}\NormalTok{(cl)}

\CommentTok{\# Convert results to a data frame}
\NormalTok{results\_df }\OtherTok{\textless{}{-}} \FunctionTok{as.data.frame}\NormalTok{(results)}
\FunctionTok{colnames}\NormalTok{(results\_df) }\OtherTok{\textless{}{-}} \FunctionTok{c}\NormalTok{(}\StringTok{"beta.LR"}\NormalTok{, }\StringTok{"SE.LR"}\NormalTok{, }\StringTok{"SE.LRHW"}\NormalTok{, }\StringTok{"SE.LRJN"}\NormalTok{, }\StringTok{"beta.PO"}\NormalTok{, }\StringTok{"SE.PO"}\NormalTok{, }\StringTok{"SE.POHW"}\NormalTok{,  }\StringTok{"SE.POJN"}\NormalTok{)}
\end{Highlighting}
\end{Shaded}

\begin{verbatim}
##   Statistic   beta.LR     SE.LR    SE.LRHW   SE.LRJN   beta.PO     SE.PO
## 1      Mean 0.4949896 0.1031785 0.09802961 0.1063741 0.4949896 0.1005392
##      SE.POHW   SE.POJN
## 1 0.09802961 0.1021607
\end{verbatim}

Now that p/n can be considered small (although usually we would want
p/n\textless0.1), the s.e. derived from the PO model are very similar to
the usual s.e. and also to the monte carlo s.d. This little exercise
clearly shows the problem of using inference when using PO if p/n is
big.

\end{document}
